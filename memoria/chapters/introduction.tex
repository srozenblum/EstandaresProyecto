\chapter{Introducción}{\label{ch:intro}}

\section{Marco teórico}

En el campo de la bioinformática, la gestión eficiente y el análisis de grandes volúmenes de datos genómicos y clínicos son fundamentales para avanzar en la comprensión y el tratamiento de enfermedades como el cáncer. Los investigadores y clínicos suelen trabajar con información que proviene de múltiples fuentes: historiales clínicos de pacientes, análisis de muestras biológicas, secuenciación genómica, anotaciones moleculares y bases de datos de conocimiento oncológico. Esta diversidad de datos, junto con su naturaleza altamente interconectada y jerárquica, plantea importantes limitaciones para los sistemas de bases de datos relacionales tradicionales. Este proyecto se centra en el diseño y la implementación de una base de datos NoSQL, utilizando MongoDB, para albergar y organizar los datos del estudio \textit{Acral Melanoma (TGEN, Genome Res 2017)}. Este conjunto de datos, disponible a través del cBioPortal for Cancer Genomics, ofrece una rica fuente de información genómica y clínica de pacientes con melanoma acral, un subtipo raro y agresivo de melanoma (cáncer de piel).

Dicho repositorio ofrece un conjunto de datos rico y representativo que incluye información clínica detallada de pacientes, características de muestras tumorales, y miles de variantes genómicas identificadas mediante secuenciación del exoma completo. Este tipo de datos biomédicos presenta características únicas que requieren soluciones tecnológicas específicas:

\begin{itemize}
	\item \textbf{Estructura altamente anidada}: la información clínica de un paciente contiene múltiples niveles de detalle (datos demográficos, historial de tratamientos, información de seguimiento), cada uno con su propia estructura interna.
	\item \textbf{Relaciones complejas entre entidades}: un paciente puede tener múltiples muestras, cada muestra puede contener miles de variantes, y cada variante afecta a genes específicos con implicaciones clínicas conocidas.
	\item \textbf{Necesidad de enriquecimiento continuo}: los datos genómicos requieren integración con bases de datos externas (como OncoKB para significancia clínica de mutaciones, o UniProt para información proteica) que evolucionan constantemente.
	\item\textbf{ Consultas multidimensionales}: los investigadores necesitan realizar análisis que cruzan información de pacientes, muestras y variantes de forma simultánea.
\end{itemize}

\section{Objetivos del proyecto}

Este proyecto aborda la problemática descrita mediante el diseño e implementación de un sistema integrado de gestión de datos clínicos y genómicos, aplicando tres tecnologías complementarias de estándares de datos abiertos e integración:

\begin{enumerate}
	\item \textbf{Bases de datos NoSQL (MongoDB)} para el almacenamiento flexible y escalable de datos biomédicos jerárquicos.
	\item \textbf{Tecnologías de transformación semántica (XML/XSLT) }para la generación automática de reportes clínicos visuales.
	\item \textbf{Web Semántica y ontologías (OWL/SPARQL)} para la representación formal del conocimiento y consultas avanzadas.
\end{enumerate}

El objetivo principal es demostrar cómo estas tecnologías, habitualmente estudiadas de forma aislada, pueden integrarse en un flujo de trabajo completo que va desde la captura de datos crudos hasta la consulta semántica del conocimiento biomédico. Específicamente, el proyecto implementa:

\begin{itemize}
	\item Un \textbf{modelo de datos NoSQL} con tres colecciones principales (\texttt{patients}, \texttt{samples}, \texttt{variants}) más dos colecciones de enriquecimiento externo (\texttt{oncokb\_genes}, \texttt{uniprot}), cada una con estructuras anidadas de hasta tres niveles que capturan la complejidad inherente a los datos clínico-genómicos.
	\item Un \textbf{pipeline ETL automatizado} que limpia, reestructura y enriquece los datos originales del estudio de melanoma acral, integrándolos con información actualizada de APIs públicas de relevancia oncológica.
	\item Un \textbf{sistema de generación de reportes} que consulta la base de datos MongoDB, transforma los resultados a XML y aplica plantillas XSLT para producir dashboards HTML interactivos que faciliten la visualización de relaciones complejas entre pacientes, muestras y variantes.
	\item Una \textbf{ontología OWL formal} diseñada en Protégé que modela todo el dominio del conocimiento clínico-genómico representado en la base de datos, incluyendo clases, propiedades de objeto, propiedades de datos y restricciones.
	\item \textbf{Capacidades de razonamiento automático} mediante reasoners como HermiT, que permiten inferir nuevo conocimiento (por ejemplo, clasificar automáticamente pacientes según su estado clínico o identificar muestras metastásicas).
	\item \textbf{Consultas SPARQL avanzadas} que explotan la semántica de la ontología para realizar búsquedas que serían difíciles o imposibles en bases de datos tradicionales
	\item \textbf{Generación automática de grafos RDF} a partir de los datos almacenados en MongoDB, creando un puente entre el almacenamiento NoSQL y la representación semántica.
\end{itemize}

\section{Relevancia en bioinformática}

La aproximación presentada en este trabajo es particularmente relevante para la bioinformática moderna por varias razones:

\textbf{Gestión de heterogeneidad}: Las bases de datos NoSQL permiten almacenar datos biomédicos sin forzarlos a esquemas rígidos predefinidos. Esto es crucial en investigación oncológica, donde nuevos biomarcadores, tratamientos o metodologías de secuenciación pueden requerir modificaciones frecuentes del modelo de datos.

\textbf{Trazabilidad y reproducibilidad}: La transformación de datos mediante tecnologías estándar como XML/XSLT garantiza que los reportes clínicos sean generados de forma determinista y auditab, esencial para la validación de resultados de investigación y para cumplir con regulaciones de datos clínicos.

\textbf{Interoperabilidad semántica}: El uso de ontologías OWL permite que diferentes sistemas y bases de datos biomédicas puedan compartir e integrar conocimiento de forma inequívoca. Un paciente clasificado como \texttt{PacienteFallecido} en nuestra ontología puede ser automáticamente reconocido y procesado por otros sistemas que utilicen estándares ontológicos compatibles.

\textbf{Consultas basadas en conocimiento}: SPARQL permite formular preguntas complejas que van más allá de la simple recuperación de datos. Por ejemplo, "encontrar todos los pacientes con variantes oncogénicas en genes asociados a resistencia terapéutica que además presentaron recurrencia de la enfermedad" es una consulta que explota tanto los datos como el conocimiento representado en la ontología.

\textbf{Escalabilidad hacia medicina personalizada}: La arquitectura propuesta sienta las bases para sistemas más complejos de apoyo a la decisión clínica, donde la integración de datos genómicos, clínicos y de bases de conocimiento externas es fundamental para identificar estrategias terapéuticas personalizadas.

\section{Estructura del documento}

Este documento está organizado de la siguiente manera:

\begin{itemize}

\item El Capítulo 1 introduce el contexto general del trabajo, exponiendo la motivación, los objetivos, la estructura
del documento y las tecnologías empleadas.

\item El Capítulo 2 describe la metodología empleada, detallando el diseño de la base de datos MongoDB, el pipeline ETL y de enriquecimiento, el sistema de generación de reportes mediante XML/XSLT, el modelado ontológico en Protégé, y la implementación de los scripts de generación de grafos RDF y ejecución de consultas SPARQL.

\item El Capítulo 3 presenta los resultados obtenidos, incluyendo estadísticas descriptivas de los datos almacenados, ejemplos de reportes HTML generados, la ontología resultante con sus inferencias, y los resultados de las consultas SPARQL sobre el grafo RDF.

\item El Capítulo 4 discute las conclusiones del trabajo, reflexiona sobre las limitaciones encontradas, y propone líneas de trabajo futuro para extender este sistema hacia aplicaciones clínicas reales y su integración con otras fuentes de datos biomédicos.

\item Finalmente, el Capítulo 5 expone las conclusiones del trabajo, incluyendo los aprendizajes
alcanzados, las limitaciones enfrentadas y posibles direcciones a seguir en trabajos futuros dentro de esta línea.

\end{itemize}



