\chapter{Results}{\label{ch:results}}

\subsection{Sistema de generación automática de reportes (Tarea 2)}

En esta sección se muestran los resultados obtenidos tras la ejecución del script genérico desarrollado en la Tarea 2. El sistema procesa consultas MongoDB, genera documentos XML intermedios y aplica una plantilla XSLT para producir vistas HTML finales.

La ejecución se realizó mediante el siguiente comando:

\begin{verbatim}
	python mongoxml_to_html.py \
	--uri <URI_MONGO> \
	--db EstadaresProyecto \
	--queries queries.txt \
	--xslt template.xslt \
	--outdir resultados/mongo_a_html
\end{verbatim}

\subsubsection{Resultados XML}

Tras ejecutar el script, se generan automáticamente documentos XML
correspondientes a cada consulta definida en el archivo \texttt{queries.txt}.
La Figura~\ref{fig:xml_resultado} muestra un ejemplo del XML generado.

\begin{figure}[H]
	\centering
	\includegraphics[width=0.6\textwidth]{figs/xml_ejemplo.png}
	\caption{Ejemplo de documento XML generado a partir de una consulta MongoDB.}
	\label{fig:xml_resultado}
\end{figure}

\subsubsection{Resultados HTML}

Posteriormente, cada documento XML es transformado mediante la plantilla XSLT,
obteniéndose un archivo HTML con una visualización estructurada y navegable.
La Figura~\ref{fig:html_resultado} muestra un ejemplo del resultado final.

\begin{figure}[H]
	\centering
	\includegraphics[width=0.95\textwidth]{figs/html_ejemplo.png}
	\caption{Vista HTML generada automáticamente a partir del XML mediante XSLT.}
	\label{fig:html_resultado}
\end{figure}



El sistema se probó con tres consultas diferentes, todas ellas con estructuras
de salida distintas. En todos los casos se generaron correctamente los
documentos XML y HTML, demostrando la capacidad genérica del script para
adaptarse a distintas consultas sin necesidad de modificar el código.
